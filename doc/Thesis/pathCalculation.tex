\section{Introduction}
A autonomous vehicle has as a main function to control all actors on its own.
This means that no human give a sign to the car that now its the time to turn the motor on.
To decide how the single components have to act at which time first the car need information about its own position which are as good as possible.
However, the next thing is to compute a path based on different informations.
For this problem there exist a big amount of solutions because for each type of vehicle there exist again a big amount of different algorithms to compute a path to another point.

\section{Position Calculation}


\subsection{Introduction}
One very important and critical part by the implementation of an autonomous vehicle is the calculation of the current position because there exist a big amount of problems which could not be solved.
This chapter contains on the one hand the different problems by creating a position calculating system and on the other one it describes methods to compute the best possible position.


\subsection{Position Specification}
One important part by calculating the current position is to specify how the current position have to be stored.
For this task there exist several types of structures.


\subsubsection{Cartesian Coordinate System}
The usual thing to storage a position is to storage the coordinates of the Cartesian coordinate system and also the angles how the object is aligned.
Also the 
This kind of storage has the advantage that is is very simple.
Also the memory and computing load efficiency is very good with this structure
The only disadvantage is that this structure doesn't contain the probability of the position and alternative positions.


\subsubsection{Probability Cloud}
Another solution is to store a cloud of points with the different possibilities that this is the real position of the vehicle.
This method to store the current position of the vehicle is much more precise than the storage as a simple point in a Cartesian Coordinate System but there are also a big amount of disadvantages.
First the microcontroller have to store a big amount of points.
This is a very critical point because the available amount of memory is limited by the memory of the used microcontroller.

Figure~\ref{fig:probabilityCloud} shows a simple example how a position can be stored in a probability cloud.
In this figure the arrow shows the position and the direction where the car is with the highest probability.

\begin{figure}
\input{tikz/probabilityCloud}
\caption{Probability Cloud}
\label{fig:probabilityCloud}
\end{figure}


%What did I mean with this?
%\subsection{Sensors based Methods}
%\sub subsection{Absolute Sensors}
%\sub subsection{Relative Sensor}
%\sub subsection{Environment Sensors}


\section{Path Calculation}


\subsection{Introduction}
The path calculation is a important part of an autonomous vehicle.
But this part is very specific for each type of vehicle.
As example vehicles with a steering mechanism like a car are not able to turn around on the same spot.
So it is possible to calculate a path which can be driven by most types of vehicles but for some types it is recommended to optimize it for the specific attributes of each type.


\subsection{Crawler Typed Vehicles}
A very easy task is it to implement a path calculation for a crawler typed vehicle.
This algorithm contains three steps.
\begin{itemize}
\item First: Rotate around it's own to the right angle
\item Second: Drive straight until the target place is reached
\item Last: Rotate to the target angle
\end{itemize}
To make the path better and rounder it is possible to calculate and use the path of a car like vehicle.


\subsection{Car Like Vehicles}
A algorithm for a vehicle with a steering mechanism like a car is much more difficult.
For such a vehicle there are much more requirements which have to be considered.
An examples for such a requirements is the driven radius.
Each vehicle have different minimal driving radius.
Another example is if and how a vehicle can drive backward.
So problems can appear if the vehicle have a trailer.

For this problem to create a good path planing algorithm there are a big amount of solutions.
In the following sections there are listed and explained some of such solutions.


\subsubsection{Path based on a circle}\label{sec:AtoBcircle}
The simplest method to calculate a path from A to B is to search a circle where the start and endpoint are part of it and the current direction is vertical to the center.
This circle is the simplest path from A to B but there are lots of disadvantages.
First the ending angle cannot be selected because it is the same as the start angle.
The next disadvantage is that the distance which have to be driven can be very long if the start angle is to large.
And for some endpoints there do not exist even a solution because the distance is near infinity.
Figure~\ref{fig:AtoBcircle} is an illustration for this method.

\begin{figure}
\makebox[\linewidth]{
\begin{tikzpicture}[line cap=round,line join=round,>=triangle 45,x=1.0cm,y=1.0cm]
\draw[->,color=black] (-0.36,0) -- (9.32,0);
\foreach \x in {,2,4,6,8}
\draw[shift={(\x,0)},color=black] (0pt,2pt) -- (0pt,-2pt) node[below] {\footnotesize $\x$};
\draw[->,color=black] (0,-0.32) -- (0,4.18);
\foreach \y in {,2,4}
\draw[shift={(0,\y)},color=black] (2pt,0pt) -- (-2pt,0pt) node[left] {\footnotesize $\y$};
\draw[color=black] (0pt,-10pt) node[right] {\footnotesize $0$};
\clip(-0.36,-0.32) rectangle (9.32,4.18);
\draw (1,1)-- (9,1);
\draw[line width=1.6pt] (4.32,5.84) -- (6.32,5.84);
\draw [shift={(5,-4.17)}] plot[domain=0.91:2.23,variable=\t]({1*6.54*cos(\t r)+0*6.54*sin(\t r)},{0*6.54*cos(\t r)+1*6.54*sin(\t r)});
\draw [->] (1,1) -- (4.16,3.45);
\begin{scriptsize}
\fill [color=qqqqff] (1,1) circle (1.5pt);
\draw[color=qqqqff] (1.14,1.28) node {$A$};
\fill [color=qqqqff] (9,1) circle (1.5pt);
\draw[color=qqqqff] (9.16,1.28) node {$B$};
\fill [color=black] (5.53,5.84) circle (1.5pt);
\draw[color=black] (6.46,6.22) node {$Abfahrtswinkel = 0.66$};
\fill [color=uququq] (4.16,3.45) circle (1.5pt);
\draw[color=uququq] (4.72,3.72) node {$direction$};
\fill [color=uququq] (5,-4.17) circle (1.5pt);
\draw[color=uququq] (5.3,-3.9) node {$C_1$};
\end{scriptsize}
\end{tikzpicture}

\begin{tikzpicture}[line cap=round,line join=round,>=triangle 45,x=1.0cm,y=1.0cm]
\draw[->,color=black] (-0.36,0) -- (9.5,0);
\foreach \x in {,2,4,6,8}
\draw[shift={(\x,0)},color=black] (0pt,2pt) -- (0pt,-2pt) node[below] {\footnotesize $\x$};
\draw[->,color=black] (0,-0.38) -- (0,5.5);
\foreach \y in {,2,4}
\draw[shift={(0,\y)},color=black] (2pt,0pt) -- (-2pt,0pt) node[left] {\footnotesize $\y$};
\draw[color=black] (0pt,-10pt) node[right] {\footnotesize $0$};
\clip(-0.36,-0.38) rectangle (9.5,5.5);
\draw (1,1)-- (9,1);
\draw[line width=1.6pt] (4.32,5.84) -- (6.32,5.84);
\draw [shift={(5,1.35)}] plot[domain=-0.09:3.23,variable=\t]({1*4.02*cos(\t r)+0*4.02*sin(\t r)},{0*4.02*cos(\t r)+1*4.02*sin(\t r)});
\draw [->] (1,1) -- (0.65,4.98);
\begin{scriptsize}
\fill [color=qqqqff] (1,1) circle (1.5pt);
\draw[color=qqqqff] (1.14,1.28) node {$A$};
\fill [color=qqqqff] (9,1) circle (1.5pt);
\draw[color=qqqqff] (9.16,1.28) node {$B$};
\fill [color=black] (5.85,5.84) circle (1.5pt);
\draw[color=black] (6.78,6.22) node {$Abfahrtswinkel = 1.66$};
\fill [color=uququq] (0.65,4.98) circle (1.5pt);
\draw[color=uququq] (1.2,5.26) node {$direction$};
\fill [color=uququq] (5,1.35) circle (1.5pt);
\draw[color=uququq] (5.3,1.64) node {$C_1$};
\end{scriptsize}
\end{tikzpicture}
}
\caption{A to B path based on a circle}
\label{fig:AtoBcircle}
\end{figure}


\subsubsection{Path based on a circle and a line}\label{sec:AtoBcs}
Another algorithm to plan a path from A to B is to start with an circle and when the target point is in front of the vehicle then it can drive straight.
If the radius of the circle is the minimal radius then this method returns the minimal path from A to B.
So this solution is much better than the solution where the path is based on a circle \ref{sec:AtoBcircle} but it is not possible to chose a own ending angle.
A very useful fact is that this solution return two possible solutions.
So the shortest solution can be selected.
The second path can also be used if the first one contains some known barriers.
Figure~\ref{fig:AtoBcs} is an illustration for this method.

\begin{figure}
%A to B Circle Straight

\makebox[\linewidth]{
\begin{tikzpicture}[line cap=round,line join=round,>=triangle 45,x=1.0cm,y=1.0cm]
\draw[->,color=black] (-0.48,0) -- (10.48,0);
\foreach \x in {,2,4,6,8,10}
\draw[shift={(\x,0)},color=black] (0pt,2pt) -- (0pt,-2pt) node[below] {\footnotesize $\x$};
\draw[->,color=black] (0,-0.48) -- (0,6.24);
\foreach \y in {,2,4,6}
\draw[shift={(0,\y)},color=black] (2pt,0pt) -- (-2pt,0pt) node[left] {\footnotesize $\y$};
\draw[color=black] (0pt,-10pt) node[right] {\footnotesize $0$};
\clip(-0.48,-0.48) rectangle (10.48,6.24);
\draw [shift={(3,3)},color=qqwuqq,fill=qqwuqq,fill opacity=0.1] (0,0) -- (0:0.6) arc (0:225:0.6) -- cycle;
\draw[color=qqwuqq] (5.6,9.32) -- (7.04,9.32);
\draw [dash pattern=on 2pt off 2pt] (3,3)-- (10,3);
\draw(11.46,9.6) -- (13.46,9.6);
\draw [shift={(1.94,4.06)},color=qqwuqq]  plot[domain=1.25:5.5,variable=\t]({1*1.5*cos(\t r)+0*1.5*sin(\t r)},{0*1.5*cos(\t r)+1*1.5*sin(\t r)});
\draw [shift={(4.06,1.94)},color=qqqqcc]  plot[domain=2.36:5.14,variable=\t]({1*1.5*cos(\t r)+0*1.5*sin(\t r)},{0*1.5*cos(\t r)+1*1.5*sin(\t r)});
\draw [color=qqwuqq] (2.41,5.49)-- (10,3);
\draw [color=qqqqcc] (4.68,0.57)-- (10,3);
\begin{scriptsize}
\fill [color=qqqqff] (3,3) circle (1.5pt);
\draw[color=qqqqff] (3.14,3.28) node {$A$};
\fill [color=qqqqff] (10,3) circle (1.5pt);
\draw[color=qqqqff] (10.16,3.28) node {$B$};
\fill [color=qqwuqq] (6.5,9.32) circle (1.5pt);
\draw[color=qqwuqq] (7.46,9.7) node {$Abfahrtswinkel = 225\textrm{\degre}$};
\draw[color=qqwuqq] (3.06,3.4) node {$225\textrm{\degre}$};
\fill [color=black] (11.76,9.6) circle (1.5pt);
\draw[color=black] (12.78,9.98) node {$MinimalerRadius = 1.5$};
\end{scriptsize}
\end{tikzpicture}
}
\caption{A to B path based on a circle and a line}
\label{fig:AtoBcs}
\end{figure}


\subsubsection{Path based on a circle, a line and another circle}
A algorithm where most criteria are considered is this one.
First the path starts like the path which is based on a circle and a line \ref{sec:AtoBcs} but at least there is another circle.
This method needs a little bit more computing power but it have a lot of advantages.
The main important advantage is that the start and the end angle can be every angle.
This means that every input returns a very good solution.
The next advantage is that this method returns eight paths which can be used.
Figure~\ref{fig:AtoBcsc} is an illustration for this method.

\begin{figure}
%A to B Circle Straight Circle

\makebox[\linewidth]{
\begin{tikzpicture}[line cap=round,line join=round,>=triangle 45,x=1.0cm,y=1.0cm]
\draw[->,color=black] (-0.32,0) -- (12.44,0);
\foreach \x in {,2,4,6,8,10,12}
\draw[shift={(\x,0)},color=black] (0pt,2pt) -- (0pt,-2pt) node[below] {\footnotesize $\x$};
\draw[->,color=black] (0,-0.36) -- (0,6.14);
\foreach \y in {,2,4,6}
\draw[shift={(0,\y)},color=black] (2pt,0pt) -- (-2pt,0pt) node[left] {\footnotesize $\y$};
\draw[color=black] (0pt,-10pt) node[right] {\footnotesize $0$};
\clip(-0.32,-0.36) rectangle (12.44,6.14);
\draw [shift={(3,3)},color=qqwuqq,fill=qqwuqq,fill opacity=0.1] (0,0) -- (0:0.6) arc (0:250:0.6) -- cycle;
\draw [shift={(10,3)},color=qqwuqq,fill=qqwuqq,fill opacity=0.1] (0,0) -- (-35:0.6) arc (-35:180:0.6) -- cycle;
\draw[color=qqwuqq] (8.34,9.54) -- (9.78,9.54);
\draw[color=qqwuqq] (5.6,9.32) -- (7.04,9.32);
\draw [dash pattern=on 2pt off 2pt] (3,3)-- (10,3);
\draw(11.46,9.6) -- (13.46,9.6);
\draw [color=zzffqq] (1.47,5.01)-- (10.74,5.72);
\draw [shift={(1.59,3.51)},color=zzffqq]  plot[domain=1.65:5.93,variable=\t]({1*1.5*cos(\t r)+0*1.5*sin(\t r)},{0*1.5*cos(\t r)+1*1.5*sin(\t r)});
\draw [shift={(10.86,4.23)},color=zzffqq]  plot[domain=-2.18:1.65,variable=\t]({1*1.5*cos(\t r)+0*1.5*sin(\t r)},{0*1.5*cos(\t r)+1*1.5*sin(\t r)});
\draw [color=qqqqff] (4.19,1)-- (8.92,0.29);
\draw [color=ffdxqq] (8.26,0.55)-- (2.47,4.73);
\draw [color=ffqqqq] (9.86,5.35)-- (5.41,1.37);
\draw [shift={(4.41,2.49)},color=qqqqff]  plot[domain=2.79:4.56,variable=\t]({1*1.5*cos(\t r)+0*1.5*sin(\t r)},{0*1.5*cos(\t r)+1*1.5*sin(\t r)});
\draw [shift={(9.14,1.77)},color=qqqqff]  plot[domain=-1.72:0.96,variable=\t]({1*1.5*cos(\t r)+0*1.5*sin(\t r)},{0*1.5*cos(\t r)+1*1.5*sin(\t r)});
\draw [shift={(9.14,1.77)},color=ffdxqq]  plot[domain=-2.2:0.96,variable=\t]({1*1.5*cos(\t r)+0*1.5*sin(\t r)},{0*1.5*cos(\t r)+1*1.5*sin(\t r)});
\draw [shift={(1.59,3.51)},color=ffdxqq]  plot[domain=0.95:5.93,variable=\t]({1*1.5*cos(\t r)+0*1.5*sin(\t r)},{0*1.5*cos(\t r)+1*1.5*sin(\t r)});
\draw [shift={(4.41,2.49)},color=ffqqqq]  plot[domain=2.79:5.44,variable=\t]({1*1.5*cos(\t r)+0*1.5*sin(\t r)},{0*1.5*cos(\t r)+1*1.5*sin(\t r)});
\draw [shift={(10.86,4.23)},color=ffqqqq]  plot[domain=-2.18:2.3,variable=\t]({1*1.5*cos(\t r)+0*1.5*sin(\t r)},{0*1.5*cos(\t r)+1*1.5*sin(\t r)});
\begin{scriptsize}
\fill [color=qqqqff] (3,3) circle (1.5pt);
\draw[color=qqqqff] (3.14,3.28) node {$A$};
\fill [color=qqqqff] (10,3) circle (1.5pt);
\draw[color=qqqqff] (10.16,3.28) node {$B$};
\fill [color=qqwuqq] (9.2,9.54) circle (1.5pt);
\draw[color=qqwuqq] (10.18,9.92) node {$Ankunftswinkel = 215\textrm{\degre}$};
\fill [color=qqwuqq] (6.6,9.32) circle (1.5pt);
\draw[color=qqwuqq] (7.56,9.7) node {$Abfahrtswinkel = 250\textrm{\degre}$};
\draw[color=qqwuqq] (2.98,3.36) node {$250\textrm{\degre}$};
\draw[color=qqwuqq] (10.3,3.4) node {$215\textrm{\degre}$};
\fill [color=black] (11.76,9.6) circle (1.5pt);
\draw[color=black] (12.78,9.98) node {$MinimalerRadius = 1.5$};
\end{scriptsize}
\end{tikzpicture}
}
\caption{A to B path based on two circles and a line}
\label{fig:AtoBcsc}
\end{figure}
