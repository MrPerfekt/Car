\documentclass[english, 10pt]{report}
\usepackage[T1]{fontenc}
\usepackage[latin9]{inputenc}
\usepackage{verbatim}
\usepackage{babel}
\begin{document}

\title{Diploma Thesis}


\author{Andreas Gruber}
\maketitle
\begin{abstract}
\
The amount of robots in our life rises very fast. They build our cars,
they mow the lawn, they do the hoovering and a lot of other things
which most people do not even recognize. Most of these intelligent
systems which are mobile systems, are connected to a central control
station or are remotely controlled. In some cases, however, it is
impossible to connect the mobile system to a control device or the
mobile system needs to act very fast and, therefore, needs to calculate
its behaviour by its own. In this case a device needs to know  its
current position, speed, and direction. Also a lot of other properties
need to be known to compute a behaviour which is as good as possible
for the current situation. To find out the values of these properties
the robot needs a lot of sensors. In order to keep the production
costs low, the mass of these sensors is pretty cheap which, in turn,
rises the problem that those can produce errors and they even can
brake while the device is working. In one of these cases the device
should detect that the sensor doesn't work any more and replace its
results by the result of other sensors.

The main goal of this thesis is to build a system which calculates the
most accurate current position of a model car based on a set of different
(cheap) sensors. To take more cheap sensors
has a lot of pros. First the whole device can be produced very cheaply.
Another very important consequence is that other cheap sensors help
to improve the results. The most important advantage (??) is that
errors and measuring faults can be detected and the wrong values can
be replaced by correct ones. Systems like this have a big application
spectrum in future because autonomous robots will be needed in more
and more branches.

Write about three sentences about the co-operation with FLLL in Hagenberg.\end{abstract}
\begin{itemize}
\item There is a co-operation, who is the contact person
\item What is the FLLL working in?
\item What is the general project you are working in? How does your thesis
fit into the general work of the FLLL?\end{itemize}

\tableofcontents

\chapter{Arduino}

\chapter{Datatypes}



\begin{longtable}{||c|c|r||}
Dieser Text & erscheint nur & auf der ersten Seite\\ \hline\hline
\endfirsthead
Als Kopfzeile & der anderen Seiten & wird dies gesetzt\\ \hline
\endhead
Die Fu"szeilen & f"ur & alle Seiten \\ \hline\hline
\endfoot
Nur die letzte & Fu"szeile ist etwas & BESONDERES \\ \hline\hline
\endlastfoot
Jetzt fangen & die Spalten & an\\
Hier & ist ein & Text\\
Hier & ist ein & Text\\ \hline
Hier & ist ein & Text\\
\end{longtable}


\end{document}
