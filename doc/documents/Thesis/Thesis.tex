\documentclass[11pt]{report}
\usepackage[german,english]{babel}

%\documentclass[english, 10pt]{report}
\usepackage{amssymb}
\usepackage{babel}
\usepackage{caption}
\usepackage{courier}
\usepackage{epstopdf}
\usepackage{fancyhdr}
\usepackage[T1]{fontenc}
\usepackage{geometry}                % See geometry.pdf to learn the layout options. There are lots.
\usepackage{graphicx}
\usepackage{hyperref}
%\usepackage[latin9]{inputenc}
\usepackage[utf8]{inputenc}
\usepackage{listings}
\usepackage{moreverb}
\usepackage{multirow}
%\usepackage[parfill]{parskip}    % Activate to begin paragraphs with an empty line rather than an indent
\usepackage{verbatim}
\usepackage{xcolor}


\geometry{a4paper}                   % ... or a4paper or a5paper or ... 
%\geometry{landscape}                % Activate for for rotated page geometry

%
% Listings to view code
%
 \lstset{
         basicstyle=\footnotesize\ttfamily, % Standardschrift
         numbers=left,               % Ort der Zeilennummern
         numberstyle=\tiny,          % Stil der Zeilennummern
         %stepnumber=2,               % Abstand zwischen den Zeilennummern
         numbersep=5pt,              % Abstand der Nummern zum Text
         tabsize=2,                  % Groesse von Tabs
         extendedchars=true,         %
         breaklines=true,            % Zeilen werden Umgebrochen
         keywordstyle=\color{red},
    		frame=b,         
 %        keywordstyle=[1]\textbf,    % Stil der Keywords
 %        keywordstyle=[2]\textbf,    %
 %        keywordstyle=[3]\textbf,    %
 %        keywordstyle=[4]\textbf,   \sqrt{\sqrt{}} %
         stringstyle=\color{white}\ttfamily, % Farbe der String
         showspaces=false,           % Leerzeichen anzeigen ?
         showtabs=false,             % Tabs anzeigen ?
         xleftmargin=17pt,
         framexleftmargin=17pt,
         framexrightmargin=5pt,
         framexbottommargin=4pt,
         %backgroundcolor=\color{lightgray},
         showstringspaces=false      % Leerzeichen in Strings anzeigen ?        
 }
 \lstloadlanguages{% Check Dokumentation for further languages ...
         %[Visual]Basic
         %Pascal
         %C
         C++
         %XML
         %HTML
         %Java
 }
    %\DeclareCaptionFont{blue}{\color{blue}} 

  %\captionsetup[lstlisting]{singlelinecheck=false, labelfont={blue}, textfont={blue}}
\DeclareCaptionFont{white}{\color{white}}
\DeclareCaptionFormat{listing}{\colorbox[cmyk]{0.43, 0.35, 0.35,0.01}{\parbox{\textwidth}{\hspace{15pt}#1#2#3}}}
\captionsetup[lstlisting]{format=listing,labelfont=white,textfont=white, singlelinecheck=false, margin=0pt, font={bf,footnotesize}}
%
% End Listings
%

\begin{document}

\rhead{\includegraphics[scale=.9]{picturesPublic/logoHTL.png}}
\cfoot{}
\begin{titlepage}
\thispagestyle{fancy}

\begin{center}

\vspace*{8em}

{\LARGE Diploma Thesis}

\vspace{2em}

{\large H\"ohere Technische Bundeslehranstalt Leonding \\[.5em]
Abteilung f\"ur EDV und Organisation}

\vspace{8em}

{\Huge High Level Movement Control for Autonomous Vehicle}
\end{center}

\vspace{18em}

Submitted by: {\bf Andreas Gruber, 5BHDVK} \\[.5em]

Date: {\bf \today} \\[.5em]

Supervisor: {\bf Peter Bauer}
\end{titlepage}
\section*{Declaration of Academic Honesty}
Hereby, I declare that I have composed the presented paper independently on my own and without any other resources than the ones indicated. All thoughts taken directly or indirectly from external sources are properly denoted as such.

This paper has neither been previously submitted to another authority nor has it been published yet. \\[1em]
Leonding, October 6, 2012 \\[5em]
Andreas Gruber \\[5em]

%\begin{otherlanguage}{german}
\section*{Eidesstattliche Erklärung}
Hiermit erkläre ich an Eides Statt, dass ich die vorgelegte Diplom- / Bachelor- / Masterarbeit selbstständig und ohne Benutzung anderer als der angegebenen Hilfsmittel angefertigt habe. Gedanken, die aus fremden Quellen direkt oder indirekt übernommen wurden, sind als solche gekennzeichnet.

Die Arbeit wurde bisher in gleicher oder ähnlicher Weise keiner anderen Prüfungsbehörde vorgelegt und auch noch nicht veröffentlicht. \\[1em]
Leonding, am 6. Oktober 2012 \\[5em]
Andreas Gruber
%\end{otherlanguage}
\begin{abstract}
The amount of robots in our life rises very fast.
They build our cars, they mow the lawn, they do the hoovering and a lot of other things which most people do not even recognize.
Most of these intelligent systems which are mobile systems, are connected to a central control
station or are remotely controlled. In some cases, however, it is
impossible to connect the mobile system to a control device or the
mobile system needs to act very fast and, therefore, needs to calculate
its behaviour by its own. In this case a device needs to know  its
current position, speed, and direction. Also a lot of other properties
need to be known to compute a behaviour which is as good as possible
for the current situation. To find out the values of these properties
the robot needs a lot of sensors. In order to keep the production
costs low, the mass of these sensors is pretty cheap which, in turn,
rises the problem that those can produce errors and they even can
brake while the device is working. In one of these cases the device
should detect that the sensor doesn't work any more and replace its
results by the result of other sensors.

The main goal of this thesis is to build a system which calculates the
most accurate current position of a model car based on a set of different
(cheap) sensors. To take more cheap sensors has a lot of pros. First the whole device can be produced very cheaply.
Another very important consequence is that other cheap sensors help
to improve the results. The most important advantage is that %This can be improved
errors and measuring faults can be detected and the wrong values can
be replaced by correct ones. Systems like this have a big application
spectrum in future because autonomous robots will be needed in more
and more branches.
\end{abstract}

%Write about three sentences about the co-operation with FLLL in Hagenberg.
%\begin{itemize}
%\item There is a co-operation, who is the contact person
%\item What is the FLLL working in?
%\item What is the general project you are working in? How does your thesis
%fit into the general work of the FLLL?\end{itemize}

\begin{otherlanguage}{german}
\begin{abstract}

\end{abstract}
\end{otherlanguage}
\tableofcontents
\section{Introduction} \label{sec:environmentIntroduction}
In this chapter the setup of the autonomous vehicle is presented. 
Here all microcontroller which were cousidered to be used are listed.
Also the main usage of most of them is shown.
Another topic of this chapter is an introduction to the environment of Arduino.
Last but not least the different types which were used in this project of sensors are shown.


\section{AVR} \label{sec:AVR}
Throughout this section all informations and pictures are taken from \cite{web:Atmel}, otherwise it is described separately.

AVR is the name of a microcontroller-family which is produced by the company Atmel.
Atmel currently produces different types of microcontrollers.
These types can be grouped in:

\begin{itemize}
\item 32-bit AVR UC3
\item AVR XMEGA
\item Automotive AVR
\item megaAVR
\item tinyAVR
\end{itemize}

These families differ in size of memory, clock rate and the supply voltage which is needed by the controller.
As an example the tinyAVR also known as ATtiny is a very small and cheap microcontroller which only needs a power supply of 0,7 V.
Beside these listed chips there are other ones which have the function to handle the battery management of Li-Ion batteries.
In addition to AVR Atmel also produces other chip-families.
For details we refer to \url{ http://www.atmel.com/products/microcontrollers/avr/default.aspx }

For this thesis mostly megaAVR also known ATmega are used because these are the most popular ones and so they are used on the Arduino boards.
These boards are shown in \ref{sec:arduino}.


\subsection{ATmega~2560} \label{sec:atmega2560}
In this section all informations and pictures are taken from \cite{manual:atmega2560}.

We start with a brief description of the AVR microcontroller which is used on the Arduino~Mega~2560 boards.
These boards will be shown in \ref{sec:arduino}.
ATmega~2560 is part of many different electronic projects.
It is a very many-sided microcontroller because it has has a lot of different features implemented in hardware.
This chip contains a lot of input and output interfaces so it can communicate with other chips without wasting invaluable computing time.
Another advantage of this these hardware implemented I/O interfaces is that no code has to be written .

In the following the main technical features will be summarized:


\subsubsection{Summary} \label{sec:atmega2560Summary}
\begin{tabular}{ll}
Speed Grade	& 0-16 MHz	\\
Flash Memory	& 256 KB	\\
SRAM			& 8 KB	\\
EEPROM		& 4 KB	\\
I/O Lines		& 86		\\
8-bit Timer		& 2		\\
16-bit Timer	& 4		\\
4-bit PWM		& 4		\\
2 to 16-bit PWM	& 12		\\
10-bit ADC		& 16		\\
USART / UART	& 4		\\
TWI / I2C		& 1		\\
\end{tabular}


\section{Arduino} \label {sec:arduino}
In this section all information and pictures are taken from \cite{web:arduino} otherwise it is described separately.

The term Arduino stands for a full environment to program different microcontrollers.
All parts of the Arduino environment are a open source. 
The environment is offered with different boards which vary in size, features, price, ... .
On each board there is the full hardware to program and run the microcontroller.
So a USB to serial tunneler is installed on-board.
Most boards also contain a voltage regulator to feed it by an external power supply too and a circuit to switch to the USB power supply.
At least each board contain a button to reset the microcontroller.

Another Part of the Arduino environment is the Arduino software.
With this software it is possible to compile C++ code with the avr-gcc compiler which is part of the GNU Compiler Collection.
This software can also transfer the compiled sketch to the USB to serial converter and this chip programs the microcontroller.

A further part of the Arduino environment is the library.
This library contains code to address the pins.
In result of that the sketch code is more compact and therefore easier to read.


\subsection{Arduino Shields} \label{sec:arduinoShields}
Shields are boards which have the design that they can be plugged on top of the Arduino.
Those shields are used to extend the Arduino capabilities.
Most shields are designed for the Arduino~UNO but the design of the Arduino~Mega is compatible for most of the shields of the Arduino~UNO.
But the shields of the Arduino~Mega does not fit on an Arduino~UNO because it has much more pins and functions than the Arduino~UNO.

Examples for Arduino Shields are:
\begin{itemize}
\item MotorDriver Shields
\item GPS Shields
\item Memory Shields
\item Display Schields
\end{itemize}


\subsection{Arduino~Mega~2560} \label {sec:arduinoMega2560}
\begin{figure}
\includegraphics[scale=0.5]{picturesArduino/arduinoMega2560_R3_Front.jpg}
\includegraphics[scale=0.5]{picturesArduino/arduinoMega2560_R3_Back.jpg}
\caption{Board of the Arduino Mega 2560}
\label{fig:mega2560}
\end{figure}

The Arduino~Mega~2560 is a microcontroller board based on the ATmega~2560.
Basically this is a board holding an ATmega~2560 so the technical features from the ATmega~2560 are also applicable for the Arduino~Mega~2560.
See figure~\ref{fig:mega2560} for an image of this microcontroller.



\subsection{Datatypes} \label {sec:datatypes}
Here the data types of AVRs are described because they differ significantly from the sizes normally used on ``regular'' PCs.
One example is the type int which only has a size of 2 bytes instead of 4 bytes like on an PC.
Another very interesting type is floating-point numbers. 
On AVRs floating-point calculations have some significant characteristics.
A very interesting fact is that floating-point calculations are much slower than integer calculations in relation to regular CPUs because AVRs have no floating point unit.
This means that floating-point calculations have to be emulated by other operations. Also the precision of floating-point calculations differs from usual PCs.
On a PC a float has a size of 4 bytes and a double has a size of 8 bytes.
On AVRs a double is as large as a float and has 4 bytes. As a consequence the usual precision is six to seven digits.
See table~\ref{tab:datatypes} for details.

\begin{table}
{\large
\begin{center}
\begin{tabular}{|l|r|r|r|}
\hline
\multirow{2}{*}{data type} & \multirow{2}{*}{Bit} & \multicolumn{2}{|c|}{value}  \\
\cline{3-4}
& & min & max \\
\hline
int8\_t, signed char				& $8$	& $-2^{7}$		& $2^{7}-1$	\\
int16\_t, signed short, signed int		& $16$	& $-2^{15}$		& $2^{15}-1$	\\
int32\_t, signed long				& $32$ 	& $-2^{31}$		& $2^{31}-1$	\\
int64\_t, signed long long			& $64$ 	& $-2^{63}$		& $2^{63}-1$	\\
& & & \\
uint8\_t, unsigned char			& $8$	& $0$			& $2^{8}-1$	\\
uint16\_t, unsigned short, unsigned int	& $16$ 	& $0$ 			& $2^{16}-1$	\\
uint32\_t, unsigned long			& $32$ 	& $0$			& $2^{32}-1$	\\
uint64\_t, unsigned long long		& $64$ 	& $0$			& $2^{64}-1$	\\
& & & \\
float, double					& $32$ 	& $-3.4028235*10^{38}$	& $3.4028235*10^{38}$	\\
\hline
\end{tabular}
\end{center}
}
\caption{Sizes of data types on the AVR}
\label{tab:datatypes}
\end{table}



\subsection{Microcontroller I/O} \label{sec:microcontrollerIO}
In this section the interface to the Arduino I/O is described.
First the basic analog and digital I/O is shown.
This is followed by a description of the UART API. 
Finally the I2C is described.
For the more complex parts of the I/O we give small code snippets to illustrate the programming of the I/O!


\subsubsection{Input Pins:} \label{sec:inputPins}
The following instruction describes how a value can be read from a pin.
\begin{itemize}
\item First the pin has to be brought into input mode.\\
\lstinline|pinMode(pin, INPUT);|

\item Then the internal pull up resistor can either be switched on or off.\\
Turn on:\\
\lstinline|digitalWrite(pin, HIGH);|\\
Turn off:\\
\lstinline|digitalWrite(pin, LOW);|\\

\item If the microcontroller has the right settings the pin can be read.
One possibility is to read a digital value.
This returns False if the voltage is lower than 2 volts and True if the voltage is above 3 volts.
If the voltage is between 2 and 3 volts either True or False can be returned.
In this case the result can be random.\\
\lstinline|value = digitalRead(pin);|\\
The other possibility is to read an analog value.
This returns a value with a resolution of 10 Bit.
So it will return a value between 0 and 1024.
0 stand for 0V and 1024 stand for 5V.\\
\lstinline|value = analogRead(Apin)|\\
\end{itemize}


\subsubsection{Digital Output Pins}\label{sec:digitalOutputPins}
The digital output of the Arduino ports can be used to send data or switch some components off and on.
Very interesting is that the the high output voltage is lower than the supply voltage and the low output voltage is higher than the supply voltage.
This is because the transistors inside the IC have a small voltage drop.
Before the digital output pins can be used the pinmode has to be set to output:\\
\lstinline|pinMode(pin, OUTPUT);|\\
To write a low voltage on the port this code line have to be executed:\\
\lstinline|digitalWrite(pin, LOW);|\\
This code line write a high voltage on the port:\\
\lstinline|digitalWrite(pin, HIGH);|\\



\subsubsection{Analog Output Pins - PWM}\label{sec:analogOutputPinsPWM}
\begin{figure}
\makebox[\linewidth]{
\includegraphics[scale=0.5]{picturesArduino/pwm.jpg}
}
\caption{Schematic of PWM}
\label{fig:pwm}
\end{figure}

PWM, which stand for pulse width modulation, is one of many methods to produce an analog voltage.
The main thing to know that PWM in the strict sense doesn't make a real analog voltage.
In truth PWM switches the pin frequently on and off. 
Many electronic components allow to give the required input via PWM.
If a real analog signal is required smoothing capacitors can be used.
This capacitors work as low pass filter and return a real analog voltage.
For an illustration of PWM see figure~\ref{fig:pwm}. 
Example for electronic components which can be controlled with PWM and their usage are:
\begin{itemize}
\item Led
	\subitem Car's brake light
	\subitem Background lighting of displays
\item Motor driver
	\subitem CPU air cooler
	\subitem Speed control of model cars
\item Servo
	\subitem Steering control of model cars
	\subitem Arms of industrial robots
\item Smoothing capacitors
	\subitem Data transfer
	\subitem Reference voltage for sensors

\end{itemize}

The resolution of the PWM pins are 8 bit so the minimal value is 0 which equals to the low level voltage and the maximal value is 255 which equals to the high level voltage. 
To output a PWM signal the port has to be set as output. 
A general description can be found in the section~\ref{sec:digitalOutputPins}.
Then the following code line can be executed.\\
\lstinline|analogWrite(PWMpin, value);|



\subsubsection{UART}\label{sec:uart}
The term UART means Universal Asynchronous Receiver Transmitter also known as RS-232.
UART is a serial interface and was one of the most important interfaces on PCs until it got replaced by USB. On microcontrollers it still plays an important role nowadays.
To use the RS-232 interface on newer PCs it is possible to tunnel it trough USB or Bluetooth. 
The OS of the PC creates a virtual RS-232 interface.
To create a tunnel of a RS-232 interface above USB it is possible to use an IC which provides such functionality.
Another possibility is to completely emulate the RS-232 tunnel via software.
This is also the method which is used on the Arduino.

To use the UART on the Arduino the following code has to be executed.\\
Before the serial connection is used it has to be started: \\
\lstinline|Serial.beginn(speed);| \\
Then data which have a size of one byte can be written to the serial port.
If there is more than one byte this method has to be called for each byte: \\
\lstinline|Serial.write(value);| \\
Or something can be printed to the serial port. 
This can be used instead of Serial.write:\\
\lstinline|Serial.println(value);| \\

A small code snippet for a serial connection could look like this:\\
\begin{lstlisting}
Serial.begin(9600);
Serial.println("Hello World");
byte in;
while(Serial.available() > 0)
	in = Serial.read();
char c = Serial.read();
\end{lstlisting}



\subsubsection{I2C}\label{sec:i2c}
The I2C bus is also a serial bus like UART with the difference that I2C uses a master and slave administration.
Another difference is, that more then two devices can be joined to the I2C bus.
The address has 7 bits and so it can be set from 0 to 127.
This result in a maximal number of devices of 128.

To use the I2C bus on the Arduino the following code has to be executed.\\
At first the I2C bus has to be started: \\
\lstinline|Wire.beginn(address);| \\
Then a request can be started: \\
\lstinline|Wire.requestFrom(device,size);| \\
At least one byte can be read from the device.
If the manual of the device say that there is more than one byte this method can be executed more than once. \\
\lstinline|byte b = Wire.read();| \\
Another option is to start a transmission:\
\lstinline|Wire.beginTransmission(device);| \\
Then it is possible to write data to the communication partner:\\
\lstinline|Wire.write(value);|\\
After sending the data the transmmission have to be closed:\\
\lstinline|Wire.endTransmission();|\\
An example for an I2C connection could look like this:\\
\begin{lstlisting}
Wire.begin()
Wire.requestFrom(0x23,1);
byte b = Wire.read();
Wire.beginTransmission(0x23);
Wire.write(b);
Wire.endTransmission();
\end{lstlisting}


\section{Electronic Components and Circuits}\label{sec:electronicCircuits}
This section contains explanations to different electronic components and circuits which were used during this thesis.


\subsection{Operational Amplifier}\label{sec:operationalAmplifier}


\subsection{Schmitt Trigger}\label{sec:schmittTrigger}
A schmitt-trigger can be build with an resistor network to an operating amplifier.
Schmitt-triggers with a constant hysteresis can also be bought as an IC.
The hysteresis describes the point in voltage where the schmitt trigger switches from logical high to logical low.


\subsection{H-Bridge}\label{sec:hBridge}


\section{Mathematic Algorithms}\label{sec:mathematicAlgorithms}





\chapter{Positioning}\label{chap:positioning}
\section{Introduction}\label{sec:introduction}
A autonomous vehicle has as a main function to control all actors on its own.
This means that no human give a sign to the car that now its the time to turn the motor on.
To decide how the single components have to act at which time 
the car need information about its own position which are as good as possible.
This chapter will present some sensors which can be used to get information about the position and rate them in different categories.
Another part of this chapter is to calculate the most possible current position based on the input data of different sensors.

\section{Sensor Types}
\begin{itemize}
\item absolute sensors
	\subitem lateration
		\subsubitem GPS	
\item environment detection
	\subitem image recognition
	\subitem bumper
	\subitem distance sensor
		\subsubitem ultrasonic
		\subsubitem infrared sensor
\item relative sensors
	\subitem touchless
		\subsubitem gyro sensor
		\subsubitem acceleration
		\subsubitem magnetometer
	\subitem other
		\subsubitem rotary encoder
		\subsubitem mouse sensor
\end{itemize}

\section{Tolerance and errors}




\bibliography{bibliography}{}
\bibliographystyle{plain}

\end{document}