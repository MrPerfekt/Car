\usepackage[german,english]{babel}

%\usepackage{algorithmic} %program
%\usepackage{amsmath}
\usepackage{amssymb}
\usepackage{babel}
%%\usepackage{blindtext}
\usepackage{booktabs}
\usepackage{calc} 
\usepackage{cancel}
\usepackage{caption}
\usepackage{circuitikz}
\usepackage{courier}
\usepackage{dcolumn}
\usepackage{epstopdf}
%\usepackage{etoolbox}
\usepackage[left]{eurosym}
\usepackage{fancyhdr}
\usepackage[T1]{fontenc}
\usepackage{forloop}
\usepackage{fp} %\usepackage{l3fp}
\usepackage{geometry}                % See geometry.pdf to learn the layout options. There are lots.
\usepackage{graphicx}
\usepackage{hyperref}
\usepackage{ifthen}
%\usepackage[latin9]{inputenc}
\usepackage[utf8]{inputenc}
%\usepackage{intcalc}
\usepackage{lcg}
%%\usepackage{lipsum}
\usepackage{listings}
%\usepackage{lmodern}
\usepackage{longtable}
%\usepackage{makeidx}
\usepackage{mathpazo}
%\usepackage{mathtools}
\usepackage{moreverb}
\usepackage{multirow}
%\usepackage[parfill]{parskip}    % Activate to begin paragraphs with an empty line rather than an indent
\usepackage{pgf}% Portable Graphics Formant
%\usepackage{pgfmath}
\usepackage{pgfplots}
%\usepackage{program} % algorithmic
\usepackage{rotating}
\usepackage{sagetex}
\usepackage{savesym}
\usepackage{tikz}%Tiks ist kein Zeichenprogram
%\usepackage{ulem}%underlinde ....
\usepackage{verbatim}
\usepackage{xcolor}

%
%Geometry
%
\geometry{a4paper}                   % ... or a4paper or a5paper or ... 
%\geometry{landscape}                % Activate for for rotated page geometry

%
%Geogebra Tikz
%
\usetikzlibrary{arrows}
\newcommand{\degre}{\ensuremath{^\circ}}

%
% Leaning Labels
%

\newcommand*{\zerobox}[2][l]{\raisebox{0pt}[0pt][0pt]{\makebox[0pt][#1]{#2}}}

\newcommand*{\slantedLabel}[1]{\rotatebox{60}{\zerobox{\bfseries #1}}}

\newcommand*{\leaningLabel}[1]{\slantedLabel{\zerobox{#1}}}

\newcommand*{\leaningLabelB}[1]{\multicolumn{1}{c}{\leaningLabel{#1}}}

%
% Listings to view code
%
 \lstset{
         basicstyle=\footnotesize\ttfamily, % Standardschrift
         numbers=left,               % Ort der Zeilennummern
         numberstyle=\tiny,          % Stil der Zeilennummern
         %stepnumber=2,               % Abstand zwischen den Zeilennummern
         numbersep=5pt,              % Abstand der Nummern zum Text
         tabsize=2,                  % Groesse von Tabs
         extendedchars=true,         %
         breaklines=true,            % Zeilen werden Umgebrochen
         keywordstyle=\color{red},
    		frame=b,         
 %        keywordstyle=[1]\textbf,    % Stil der Keywords
 %        keywordstyle=[2]\textbf,    %
 %        keywordstyle=[3]\textbf,    %
 %        keywordstyle=[4]\textbf,   \sqrt{\sqrt{}} %
         stringstyle=\color{white}\ttfamily, % Farbe der String
         showspaces=false,           % Leerzeichen anzeigen ?
         showtabs=false,             % Tabs anzeigen ?
         xleftmargin=17pt,
         framexleftmargin=17pt,
         framexrightmargin=5pt,
         framexbottommargin=4pt,
         %backgroundcolor=\color{lightgray},
         showstringspaces=false      % Leerzeichen in Strings anzeigen ?        
 }
 \lstloadlanguages{% Check Dokumentation for further languages ...
         %[Visual]Basic
         %Pascal
         %C
         C++
         %XML
         %HTML
         %Java
 }
    %\DeclareCaptionFont{blue}{\color{blue}} 

  %\captionsetup[lstlisting]{singlelinecheck=false, labelfont={blue}, textfont={blue}}
\DeclareCaptionFont{white}{\color{white}}
\DeclareCaptionFormat{listing}{\colorbox[cmyk]{0.43, 0.35, 0.35,0.01}{\parbox{\textwidth}{\hspace{15pt}#1#2#3}}}
\captionsetup[lstlisting]{format=listing,labelfont=white,textfont=white, singlelinecheck=false, margin=0pt, font={bf,footnotesize}}
%
% End Listings
%


% Variables\newcommand{\titel}{Automation of food order in nursing homes}
\newcommand{\varTitle}{High Level Movement Control for Autonomous Vehicle}
\newcommand{\varAuthor}{Andreas Gruber}
\newcommand{\varSupervisors}{Peter Bauer, Roland Richter}
\newcommand{\varLocation}{St. Peter am Wimberg}