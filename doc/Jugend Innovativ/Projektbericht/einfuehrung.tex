\begin{abstract}
Menschen mit körperlicher Beeinträchtigung haben es im Alltag sehr schwer, vor allem jene, welche durch Schicksalsschläge im Rollstuhl sitzen müssen. Auch wenn heute bereits viele Gebäude so gebaut sind, dass sie mit dem Rollstuhl befahrbar sind, lauern immer noch viele Gefahren für RollstuhlfahrerInnen. Bei manchen Menschen ist die Beeinträchtigung so schwer ausgeprägt, dass sie selbst nicht einmal in der Lage sind, einen Rollstuhl zu bedienen. 
Genau für diese Menschen soll meine Diplomarbeit eine Erleichterung im alltäglichen Leben bringen. Das System, welches ich im Zuge meine Diplomarbeit entwickle, soll Rollstuhlfahrer vor Gefahren wie Gegenstände und Stufen schützen, indem es die Position und die Umgebung des Rollstuhls erfasst und im Gefahrenfall schützend eingreift.
Weiters ist es auch möglich, den Rollstuhl direkt von diesem System steuern zu lassen und nur durch Angabe eines Zieles zu lenken. Dies erhöht nicht nur die Sicherheit von beeinträchtigten Personen, sondern auch die Selbstständigkeit, die Unabhängigkeit und die Lebensqualität für alle Betroffenen.
\end{abstract}