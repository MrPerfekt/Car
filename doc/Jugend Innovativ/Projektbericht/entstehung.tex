\section{Projektentstehung}
\subsection{Einleitung}
Es ist Sonntag, die Sonne strahlt und in dem kleinen, idyllischen Ort St. Peter am Wimberg findet wie auch die Jahre davor das jährliche Maibaumfest statt.
Jeder ist gut gelaunt und amüsiert sich.
Die Musikkapelle spielt und die Meisten genießen zu den frischen Bratwürsten ein kühles Bier.

Beim entspannten Sitzen am Biertisch beobachte ich die Menge.
Verschiedenste Personen sitzen beisammen und unterhalten sich prächtig.
Vor allem die Unterschiede an den verschiedenen Persönlichkeiten und Personen haben meine Aufmerksamkeit erregt.
Manche sind laut, manche eher schüchtern, manche dick und andere eher sehr mager.
Auch eine Familie mit einem körperlich und geistig beeinträchtigten Kind hat sich in die Menge gesellt.

Beim Beobachten aus der Ferne sieht man wie sehr die gesamte Familie die Flucht aus dem Alltag und die Geselligkeit geniest, auch wenn die Beeinträchtigungen des Kindes immer wieder Hindernisse und Einschränkungen mit sich bringen.
So ist es zum Beispiel eine richtige Herausforderung, den breiten Rollstuhl durch die Gänge zwischen den Biertischen zu manövrieren. 
Auch das schaffen, eines Platzes am Tisch für das Kind und seinen Rollstuhl ist eine Herausforderung.


\subsection{Verbesserung der Lebensqualität durch die Technik}
Eine große Verbesserung der Situation ist, dass das Kind zumindest einen automatischen Rollstuhl hat, welcher von ihm selbst gelenkt werden kann.
Dies ist meiner Meinung nach eine große Verbesserung der Mobilität des Kindes und auch eine große Erleichterung für die Eltern.
Allerdings ist es nun um so wichtiger, dass die Eltern oder seine Betreuer, wie die Lehrer an seiner Schule, um so besser Aufpassen, wohin er mit seinem Rollstuhl fährt.

Eine kleine Unachtsamkeit seinerseits kann sehr gefährlich sein.
Zudem kommt, dass er Gefahren als Kind noch nicht so gut abschätzen kann.
Hindernisse, wie eine Stiege, ein Schwimmteich oder auch nur eine Gehsteig-Kante, können für ihn sehr sehr gefährlich werden, wenn er diese übersieht.


\subsection{Einsatz von autonomen Systemen}
Beim Nachdenken, was dem Kind und den Eltern helfen könnte, fällt mir ein kleiner Roboter ein, welcher in der langen Nacht der Forschung in Hagenberg mit dem Namen Fuzzelbot präsentiert wurde.
Mit einer Kamera und einer Open Source Bildverarbeitungssoftware fährt er am Boden und versucht, zu erkannten Verkehrstafeln, eine gespeicherte Aktion auszuführen.
Dieses Prinzip kann man natürlich auch auf einen Rollstuhl anwenden.
Dabei soll dieser nicht kleine Verkehrstafeln, sondern Plätze in seiner Umgebung erkennen und diese mittelst Sprachsteuerung aufsuchen.

Allerdings gibt es dabei große Probleme selbst beim Fahren einer 90-Grad-Kurve.
Der Roboter schaltet die Lenkung ungefähr auf Recht beginnt zu fahren und fährt nach zwei Sekunden wieder gerade aus.
Das große Problem dabei ist, dass die Leistung des Motors und somit auch die Geschwindigkeit und der Weg, welcher in diesen zwei Sekunden zurückgelegt wird, von vielen Faktoren abhängt.
Um nur einig aufzuzählen, kann man das Material des Bodens, die Batterieladung, den exakten Lenkeinschlag nennen.

Auch die Bildverarbeitung ist für viele Einsatzbereiche nicht ausreichend.
So sind die Entfernungen zu Hindernissen nur mit sehr viel Rechenaufwand und sehr ungenau zu ermitteln.
Auch die genaue Position ist nur durch Bildverarbeitung nicht immer genau berechenbar.
Also bin ich zu dem Entschluss gekommen, dass Bildverarbeitung niemals reichen kann, um einen Rollstuhl vor Gefahren zu schützen.
Allerdings gibt es auch noch andere Möglichkeiten in der Technik die Umwelt wahrzunehmen.


%\subsection{Verbesserung von autonomen Systemen durch Sensorik}
\subsection{Erkundung der Umwelt}
Möglichkeiten zur Erkundung der Umwelt bieten eine Vielzahl an Sensoren.
Infrarot- oder Ultraschall-Distanzsensoren, Kompasssensoren, Lagesensoren, Beschleunigungssensoren, Hitzesensoren und Systeme wie GPS sind nur ein Bruchteil der möglichen Verfahren, mit welchen Informationen aus der Umgebung gewonnen werden können.

Allerdings genügt es nicht nur zu wissen, das bei Sensor Nr. 1 eine Distanz von 1,3 Meter gemessen wurde.
Eine wichtige Aufgabe eines Rollstuhl Schutz- und Steuerungssystem ist es auch diese große Menge an Messdaten zu analysieren und verarbeiten.
Außerdem muss das System sich immer die aktuelle Position des Rollstuhls berechnen, um früher erkannte Hindernisse bereits im Vorfeld mit einem ausreichenden Sicherheitsabstand zu umfahren.
Aus diesem Grund muss das System die tatsächliche Position des Gegenstandes berechnen und diese in einer Umgebungs-Wissensbasis speichern.
Diese Position kann aus dem Wert des Sensors, der Position des Sensors auf dem Rollstuhl und der aktuellen Position des Rollstuhls berechnet werden.


\subsection{Berechnung der eigenen Position}
Und daher muss nicht nur die Umgebung durch Sensoren abgetastet werden, auch, und das ist der wesentlich schwierigste Teil, die eigene Position muss so gut wie möglich berechnet werden.
So ist zum Beispiel GPS selbst wenn man alle Tricks, welche bereits existieren, verwendet, kann die Position nur mit einer Abweichung von +/- 10m bestimmt werden.
Ein Trick ist zum Beispiel der Vergleich mit stationären Basisstationen oder ein anderer die Berücksichtigung der Signalstärke von Handymasten.
Also müssen zusätzlich zu GPS auch noch andere Sensoren wie zum Beispiel Drehencoder an den Rädern oder Beschleunigungssensoren montiert werden.
Auch die Installierung von Lagesensoren kann für den Betroffenen lebensrettend agieren.
So kann im Falle eines Sturzes automatisch ein Alarm in Form einer Nachricht an einen Betreuer gesendet werden.

Eine mögliche Erweiterung wäre natürlich auch Fitnessdaten wie Blutdruck, Puls, Blutzucker des Betroffenen zu Messen und im Notfall wieder ein Alarm los gesendet werden.
Um in einem Ernstfall wie zum Beispiel ein Herzinfarkt wichtige Zeit zu sparen, kann der Rollstuhl aufgrund seiner gesammelten Daten nun auch zu einer optimalen Position für die Rettungskräfte fahren.
Diese Position kann zum Beispiel die Wohnungstür sein.


\subsection{Zusammenfassung}
Ein solches System würde die Lebensqualität von Menschen, welche vom Schicksal einer körperlichen Beeinträchtigung betroffen sind, enorm steigern.
Aber nicht nur die Lebensqualität, sondern auch die Mobilität könnte von einem solchen System ganz neu definiert werden.
Besonders muss man aber betonen, dass das System keinesfalls die Funktion des Betreuers ersetzen soll, sondern den Betreuer des Betroffenen unterstützen soll.
Dadurch soll der Betreuer entlastet werden von der Tätigkeit den Betroffenen zu Fahren und zu Lenken und kann sich stattdessen vollkommen um den Beeinträchtigten kümmern.


\section{Ziel der Diplomarbeit}
Bei der ersten Besprechung mit dem externen Betreuer Roland Richter aus Hagenberg und mit meinem Betreuungslehrer haben wir als Hauptziel ein selbst überwachendes System festgelegt.
Dabei war vorerst nur geplant, dass das System selbstständig Kurven mit einem bestimmten Radius und einem bestimmten Winkel fahren kann.
Auch ein einfaches geradeaus Fahren ist hier geplant gewesen.
Das Ganze sollte allerdings selbst überwachend sein und somit, selbst wenn der Motor kurzzeitig ausfällt oder die Lenkung verbogen ist immer noch funktionieren.

Im Laufe der folgenden Meetings haben wir die Anforderungen an das System immer wieder erhöht, bis wir das endgültige Ziel dieser Diplomarbeit festgelegt haben.
Dieses Ziel war, dass das Lego Auto, welches als Testsystem für diese Arbeit agierte, selbstständig in der Lage sein sollte einen vordefinierten Punkt im Raum mit einem vordefinierten Winkel anzufahren.
Dabei sollte der Lego Rollstuhl seinen Weg immer wieder aktualisieren und im Falle einer Fehlnavigation durch bestimmte Fehler, wie einen Schlupf bei den Rädern, diese Fehlnavigation wieder ausgleichen.

Nach dieser Aufgabe, welche bereits implementiert ist und sehr gut funktioniert hat haben wir noch weitere Pläne für die Fortsetzung des Projektes geschmiedet.
Bei diesen Überlegungen zu unserem Projekt haben wir uns auf die nächsten Optimierung mit der höchsten Priorität geeinigt.
Erstens soll die Lenkung des System, mit einem verbesserten Ausgleichsalgorithmus versehen werden.
Dieser Punkt ist nach der Positionsberechnung ein wesentlicher Teil meiner Arbeit.
Hierbei erhält das Auto von den Sensoren immer wieder den realen Lenkeinschlag und muss aufgrund dieser Daten und den zuvor eingestellten Lenkeinschlag immer wieder nachkorrigieren.

Zweitens soll nun die Struktur der Software noch einmal verändert werden um Kompatibilität mit jeder Sorte an Fahrzeugen zu schaffen.
Sollte nun eine weitere Art an Fahrzeug unterstützt werden müssen nur noch einzelne Komponenten hinzugefügt werden allerdings, und das ist das Wichtigste, es muss kein vorhandener Code wieder verändert werden.
Diese Komponenten, welche für neue Fahrzeugtypen ergänzt werden müssen, beschreiben die genaue Ansteuerung der Lenkung.
Durch diese Verbesserungen werden nun verschiedenste Antriebs- und Lenksysteme an Rollstühlen optimal unterstützt auch wenn die Lenksysteme sehr unterschiedlich sind.


\section{Nötige Recherchen und Vorbereitungen}
Um diese Anforderungen, welche ein solches System stellt, optimal erfüllen zu können benötigt man vor allem ein ordentliches Grundwissen in der Programmierung.
Weil dieses System auf einem Mikrocontroller laufen soll ist es eine Voraussetzung, dass dieses System in c++ geschrieben ist.
Da wir in der Schule nur c und als objektorientierte Programmiersprachen nur c\# und Java gelernt haben ist somit zwingend notwendig gewesen, dass ich mir im Vorfeld die Besonderheiten von c++ ansehen musste.
Bei dieser genaueren Erforschung dieser Sprache war mir mein Betreuungslehrer, Peter Bauer sehr hilfreich.
Er, der ein eingefleischter c++ Programmierer ist, hat mir dabei immer wieder Tipps gegeben und neue Dinge gezeigt.
Auch bei der Strukturierung der Software haben wir gemeinsam mit Roland meinen Software Entwurf immer wieder erweitert, verfeinert und verbessert.

Weiteres war es auch noch wichtig, mir ein Grundwissen im Bereich, der Elektronik anzueignen, da ich in diesem Bereich noch sehr unerfahren war.
Ein große Hilfe war dabei die Nachbarabteilung unserer Schule welche sich darauf spezialisiert hat.
Dabei haben mir immer wieder Schüler der Elektronik-Abteilung Tipps zur Verbesserung meiner Hardware gegeben.

Als letzten Punkt hat natürlich auch die Anbindung meiner Software an den Rollstuhl untersucht werden müssen.
Hierbei haben sich ich und mein Betreuungslehrer mehrere Arten überlegt wie ein Rollstuhl motorisiert werden kann.
Auch die spezifischen Vor- und Nachteile der einzelnen Varianten haben wir getestet und versuch die Fehler durch Optimierungen zu verkleinern.


\section{Planung und Timing}
Ein wichtiger Punkt war auch die Planungsmethode welche bei diesem Projekt angewendet werden sollte.
Bei unserem ersten Treffen haben wir uns dazu entschieden die Planung etappenweise zu gestalten.
Dabei habe wir bei jedem unserer Treffen neue Möglichkeiten zur Weiterentwicklung diskutiert und diese nach Priorität gereiht.
Die wichtigsten dieser Aufgaben haben wir schließlich als mein Ziel bis zum nächsten Meeting gemacht.
Allerdings habe ich, beim Auftreten von Problemen, per E-Mail mit Roland diese Aufgaben immer wieder besprochen und, falls es nötig gewesen ist, diese Abgeändert.
Zur detaillierten Zeitplanung finden sie im Anhang eine genauere Visualisierung der Aufgaben und die zeitliche Einreihung.


Naja; musst halt ein paar Meilensteine festlegen (im Nachhinein). 
Wann hast die Arduino Plattform einigermaßen vestanden gehabt: Datum x).
In die Zielplattform eingearbeitet: Datum x. Erster Prototyp: Datum y, Feature Definition: Datum z, …
[16:25:34] Peter Bauer: . War "Kunde". Aufgabe: Definition der Requirements. Technischer Ansprechpartner.


\section{Aufgabenverteilung und Ressourcenplanung}
Die Aufgabenverteilung ist bei meinem Projekt relativ einfach gehalten.
Im Detail sieht sie so aus: \\
Institut für wissensbasierte mathematische Systeme an der JKU Linz
\begin{itemize}
\item Roland Richter \\ (Institut für wissensbasierte mathematische Systeme an der JKU Linz)
\subitem Auftraggeber
\subitem Definition der Anforderungen, Technischer Ansprechpartner
\item Peter Bauer
\subitem Betreuungs-Lehrer
\subitem Formularer und Technischer Ansprechpartner
\item Andreas Gruber
\subitem Entwickler
\end{itemize}

%\begin{tabular}{lll}
%Rolle & Name & Aufgabe \\
%\hline
%Auftraggeber & Roland Richter & Definition der Anforderungen, Technischer Ansprechpartner \\
%Betreuungs-Lehrer  & Peter Bauer & Formularer und Technischer Ansprechpartner \\
%Schüler & Andreas Gruber & Entwickler \\
%\end{tabular}