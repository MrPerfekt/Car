\section{Organisation und Durchführung}
Zuerst möchte ich anmerken, dass das Projekt eine Einzelarbeit von mir ist und es daher keine weiteren Projektmitglieder gibt.
Ich arbeite zirka seit einem dreiviertel Jahr an diesem Projekt, in dem ich mich monatlich mit meinem Betreuungslehrer und Roland Richter getroffen habe, um mein Projekt Schritt für Schritt zu planen.
Bei jedem dieser Treffen haben wir meine Arbeit des letzten Monats besprochen, mögliche Fehler analysiert und Ziele für das nächste Monat festgelegt.
Dies ist für ein Projekt meiner Art die beste Möglichkeit dies durchzuführen, da es sehr schwierig gewesen wäre das Projekt am Anfang komplett durchzuplanen, da es immer wieder Änderungen, Probleme und Fehler gegeben hat.
Diese Art das Projekt durchzuführen hat sich sehr gut bewährt und ich würde es jederzeit wieder so machen.


\section{Zusammenarbeit}
Professor Bauer ist mein Betreuungslehrer während der gesamten Projektdauer.
Egal ob ich Hilfe bei meinen Problemen oder Ideen brauchte, zu ihm kann ich immer kommen.
Er hilft mir so gut er kann, mein Projekt zu realisieren.
Wenn sich ein Problem ergibt, dass ich alleine nicht mehr lösen kann, gehe ich immer zuerst zu ihm und bitte ihn mir zu helfen.
Meistens sind wir zu zweit dann auch zu einer sehr guten Lösung gekommen.
Außerdem werden alle formalen Dinge die für meine Diplomarbeit wichtig sind über Professor Bauer geregelt.

Wenn ich und mein Betreuungslehrer uns einmal nicht einig sind oder wir bei einem Problem beide nicht mehr weiter gekommen, bitten wir Roland Richter um Hilfe.
Zu dritt haben wir es dann fast immer geschafft, eine sehr gute Lösung zu finden. 
Roland informiere ich immer per E-Mail über den aktuellen Projektstand, so ist er auch immer mitten im Geschehen drinnen und bei einem Problem war er schnell eingearbeitet in mein Projekt.
Oft holte ich mir aufgrund seiner Erfahrung auch Tipps, wie ich am besten etwas realisieren könnte, da kann er mir immer sehr gut weiter helfen.


\section{Erfahrungen}
Neben dem eigentlichen Resultat, nämlich der Software und dem Auto-Testsystem, habe ich auch sehr viele Erfahrungen durch dieses Projekt gemacht.
Ein ganz neuer Bereich war für mich der elektronische Teil dieses Systems.
Da ich vor meiner Diplomarbeit einen Lötkolben nur sehr selten verwenden habe geschweige denn eine eigene elektronische Schaltung entworfen habe war bei diesem Projekt nicht nur der Softwareteil eine große Herausforderung.
Mittlerweile bin ich aber auch in diesem Bereich sehr gut eingearbeitet und habe deshalb sehr viel davon profitiert.

Meine Erfahrungen beschränken sich aber nicht nur auf den Hardware-Bereich dieses Projekt, auch softwareseitig habe ich viel dazu gelernt.
Hierbei kann ich vor allem meine jetzigen c++ Kenntnisse anführen, welche ich ohne dieser Diplomarbeit wahrscheinlich niemals bekommen hätte.
Aber auch bei der Strukturierung der Software habe ich von meinen Betreuern sehr viel gelernt.
Sie haben mir immer Tipps gegeben und mir viele Prinzipien wie Design-Patterns gelernt.


\section{Konfliktbewältigung}
Wie schon erwähnt, treffen wir drei uns monatlich zu einer Teambesprechung. In diesen versuchen wir, auch immer auftretende Probleme zu lösen.
In vielen Fällen schaffen wir es nach einigem Diskutieren und vielen Überlegungen auf eine gute Lösung zukommen.
Einmal war es jedoch der Fall, dass wir aufgrund der Komplexität des Problems trotz mehrstündigen und intensiven Nachdenken und Diskutieren zu keiner vernünftigen Lösung kommen.
Zu diesem Problem ist uns die Lösung dann erst einige Monate später eingefallen.
Meistens war es aber der Fall, dass wenn drei, sich in dem Bereich auskennende Personen über ein Problem reden, man eine gute und sinnvolle Lösung erhält, nach dem Motto, drei Köpfe sind besser als einer.